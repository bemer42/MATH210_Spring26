\documentclass[10pt]{article}
\usepackage{fullpage}
\usepackage{amsfonts}
\usepackage{amsmath}
\usepackage{amsthm}
\usepackage{graphicx}
\usepackage{color}
\usepackage{amssymb}
\usepackage{empheq}
\usepackage{mathrsfs}
\usepackage{enumerate}
\usepackage{tikz}
\usepackage{pgflibraryarrows}
\usepackage{pgflibrarysnakes}
\usepackage{upgreek}
\usepackage{tipa}
\usepackage{multicol}
\usepackage{verbatim}
\usepackage{floatrow}
\usepackage{gensymb}
\usepackage{caption}

\usepackage{versions}
\excludeversion{sol}
\includeversion{sol}
\newenvironment{solution}{
\sol\\{\sc{Solution:}}}{
$\hfill\blacksquare$\endsol}

\def\|{||}
\def\R{\textbf{R}}
\def\C{\mathbb{C}}


\usepackage[T1]{fontenc}
\usepackage[font=small,labelfont=bf,tableposition=top]{caption}

\DeclareCaptionLabelFormat{andtable}{#1~#2  \&  \tablename~\thetable}

\newcommand{\pd}[2]{\frac{\partial #1}{\partial #2}}
\newcommand{\pdd}[2]{\frac{\partial^2 #1}{\partial {#2}^2}}
\newcommand{\pddd}[2]{\frac{\partial^3 #1}{\partial {#2}^3}}
\newcommand{\de}[2]{\frac{d #1}{d #2}}
\newcommand{\ppdd}[3]{\frac{\partial^2 #1}{\partial #2 \partial #3}}

\newtheorem*{1}{Problem 1}
\newtheorem*{2}{Problem 2}
\newtheorem*{3}{Problem 3}
\newtheorem*{4}{Problem 4}


\newfloatcommand{capbtabbox}{table}[][\FBwidth]

\usepackage{fancyhdr}
\setlength{\headheight}{15.2pt}
\pagestyle{fancy}
\setlength\headsep{30pt}
\lhead{Phase 1 Worksheet}
\rhead{Kutztown University -- MATH 210}

\fancyfoot{}


%\rfoot{\tiny\textcopyright \textcolor{gray}{Brooks Emerick}}

\begin{document}

\begin{center}
\Large{\textsc{Phase 1 (part 2): Optimization on an Interval}}\\
\end{center}




%%%%%%%%%%%%%%%%%%%%%%%%%%%%%%%%%%%%%%%%%
\begin{enumerate}[{$1.]$}]


\item \textsc{Wire Problem:}  A 50-inch piece of wire is to be cut into two pieces (a portion of length $x$ and a portion of length $50-x$, see figure below), which are then bent into a square and a circle, respectively.  Where should the wire be cut (i.e. what is $x$?) in order to minimize the sum of the areas of these two shapes?  Where should it be cut to maximize the sum of the areas? To solve both problems, formulate a function $f(x)$ that represents the sum of the two areas in terms of $x$, then use the Closed Interval Method to determine the absolute minimum (and maximum) on the interval $x\in [0, 50]$.  
 
 \begin{center}
 	\hspace{-2cm}\includegraphics[width = .765\textwidth]{P1WS2_Figure_1.png}	\vfill
\end{center}



\pagebreak

\item \textsc{Pivot Problem:}  Two perpendicular hallways, one of width 6 ft and the other of width 5 ft, meet at a right-angle corner. A rigid ladder (modeled as infinitely thin) is carried horizontally around the corner. What is the longest ladder that can make the turn?



\end{enumerate}


















\end{document}