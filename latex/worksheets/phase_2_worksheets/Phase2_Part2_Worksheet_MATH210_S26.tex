%%%%%%%%%%%%%%%%%%%%%%%%%%%%%%%%%%%%%%
% Main Preamble
%%%%%%%%%%%%%%%%%%%%%%%%%%%%%%%%%%%%%%
\documentclass[10pt]{article}
\usepackage{fullpage}
\usepackage{amsfonts}
\usepackage{amsmath}
\usepackage{amsthm}
\usepackage{graphicx}
\usepackage{color}
\usepackage{amssymb}
\usepackage{empheq}
\usepackage{mathrsfs}
\usepackage{enumerate}
\usepackage{tikz}
\usepackage{pgflibraryarrows}
\usepackage{pgflibrarysnakes}
\usepackage{upgreek}
\usepackage{tipa}
\usepackage{multicol}
\usepackage{verbatim}
\usepackage{floatrow}
\usepackage{gensymb}
\usepackage{caption}

\usepackage[T1]{fontenc}
\usepackage[font=small,labelfont=bf,tableposition=top]{caption}

\DeclareCaptionLabelFormat{andtable}{#1~#2  \&  \tablename~\thetable}

%%%%%%%%%%%%%%%%%%%%%%%%%%%%%%%%%%%%%%
% Header/footer
%%%%%%%%%%%%%%%%%%%%%%%%%%%%%%%%%%%%%%
\usepackage{fancyhdr}
\setlength{\headheight}{15.2pt}
\pagestyle{fancy}
\setlength\headsep{30pt}
\lhead{P2.WS2}
\rhead{Kutztown University -- MATH 210}
\fancyfoot{}

%%%%%%%%%%%%%%%%%%%%%%%%%%%%%%%%%%%%%%
% Document
%%%%%%%%%%%%%%%%%%%%%%%%%%%%%%%%%%%%%%
\begin{document}

\begin{center}
\Large{\textsc{Phase 2 (part 2): Polynomial Interpolation }}\\
\end{center}

\begin{enumerate}[{$1.]$}]

\item Suppose you want to interpolate the points $(-1, 0)$, $(0,1)$, $(2,0)$, $(3,1)$, and $(4,2)$ by a polynomial of as low a degree as possible. 

\begin{enumerate}[{$\quad a.)$}]
	\item What degree should you expect this polynomial to be?  \vspace{1cm}
	\item Write out a linear system of equations for the coefficients of the interpolating polynomial.  \vfill
	\item Create a Python program that solves this system numerically and report the solution below. \vspace{1cm} 
\end{enumerate}


\pagebreak
\item The table below shows the population figures (in millions) for three countries over the same 30-year period.  Plot the data and interpolating polynomial on the same graph. 

\begin{center}
\begin{tabular}{ c   c   c  c}
\hline
Year & United States & China & German \\ 
\hline
1980 & 227.225 & 984.736 & 78.298  \\ 
1990 & 249.623 & 1148.364 & 79.380 \\ 
2000 & 282.172 & 1263.638 & 82.184 \\ 
2010 & 308.282 & 1330.141 & 81.644 \\\hline \hline
\end{tabular}
\end{center}

Create a single file that outputs the following: 
\begin{enumerate}[{$\quad a.)$}]
	\item Use cubic polynomial interpolation to estimate the population of China in 1992. 
	\item Use cubic polynomial interpolation to estimate the population of the USA in 1984. 
	\item Use cubic polynomial interpolation to make plot of the German population from 1980 to 2010.  Your plot should show a smooth curve and be well annotated. 
\end{enumerate}

\end{enumerate}


\end{document}