\documentclass[10pt]{article}
\usepackage{fullpage}
\usepackage{amsfonts}
\usepackage{amsmath}
\usepackage{amsthm}
\usepackage{graphicx}
\usepackage{color}
\usepackage{amssymb}
\usepackage{empheq}
\usepackage{mathrsfs}
\usepackage{enumerate}
\usepackage{tikz}
\usepackage{pgflibraryarrows}
\usepackage{pgflibrarysnakes}
\usepackage{upgreek}
\usepackage{tipa}
\usepackage{multicol}
\usepackage{verbatim}
\usepackage{floatrow}
\usepackage{gensymb}
\usepackage{caption}


\usepackage{fancyhdr}
\setlength{\headheight}{15.2pt}
\pagestyle{fancy}
\setlength\headsep{30pt}
\lhead{MP 1 (part 2) Worksheet}
\rhead{Kutztown University -- MATH 210}

\fancyfoot{}


%\rfoot{\tiny\textcopyright \textcolor{gray}{Brooks Emerick}}

\begin{document}

\begin{center}
\Large{\textsc{Mini-Project 1 (part 2): Rectangular Optimization with Borders}}\\
\end{center}


%%%%%%%%%%%%%%%%%%%%%%%%%%%%%%%%%%%%%%%%%
\begin{enumerate}[{$1.]$}]


\item \textsc{Maximizing Area with Borders:}  Summarize the three basic cases below for maximizing area, $A$, given the perimeter, $P$.  For each case, provide the optimal dimensions, $x$ and $y$, and the maximum area, $A_{max}$, which is a function of $P$.  \\

\includegraphics[width = .2\textwidth]{MP1_Figure_1.png}


\pagebreak

\item \textsc{Minimizing Perimeter with Borders:}  Summarize the three basic cases below for minimizing perimeter, $P$, given the area, $A$.  For each case, provide the optimal dimensions, $x$ and $y$, and the minimum perimeter, $P_{min}$, which is a function of $A$.  \\

\includegraphics[width = .2\textwidth]{MP1_Figure_1.png}



\end{enumerate}


















\end{document}